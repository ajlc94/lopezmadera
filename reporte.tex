\documentclass[12pts]{article}
\usepackage[left=12mm, top=0.5in, bottom=5in]{geometry}


\begin{document}
\title{PROYECTO FINAL: INTELIGENCIA ARTIFICIAL}
\author{Alberto de Jesus Lopez Castro \and Luis Armando Madera Hau}
\maketitle

\section{introduccion}
	En este portafolio veremos la aplicación de uno de los métodos que actualmente se// utilizan para la resolución de problemas por medio de un agente y un entorno en el cual// nuestro entorno es un laberinto que se crea aleatoriamente como veremos a continuación y// nuestro agente inteligente que es capaz de salir de dicho laberinto.//
	
Se aplicaran conocimientos obtenidos en clase como la búsqueda en anchura y se mostraran// ejemplos para que el usuario sea capaz de comprender de una manera mas clara el método que// se utiilizo para la resolución de la problemática.//

\section{laberinto aleatorio}
En este programa el laberinto cuenta con una entrada y una salida estatica, es decir que//
a pesar de que el laberinto se genera aleatoriamente su entrada y su salida siempre serán// las mismas o estarán en su misma posición utilizando ciertas condiciones dentro de los// siclos que se utilizan para generar el laberinto.//

Ademas  en el cual se utilizará el carácter # representando paredes o muros y el espacio// vacio como si fueran pasillos por donde el agente el capaz de avanzar y utilizando un ramdon
para generar los muros dentro del laberinto.//

\section(busqueda en anchura)
Busqueda en anchura n algoritmo de búsqueda en anchura recorre todos los nodos de un árbol// de manera uniforme. Expande cada uno de los nodos de un nivel antes de continuar con el// siguiente.//
Expandir los nodos de forma uniforme garantiza encontrar la mejor solución de un problema //
de costo uniforme antes que ninguna, de manera que apenas se encuentre una solución, puede// devolverse, porque será la mejor, o bien expandir todo el nivel en el cual se hubiere// encontrado, para obtener todas las soluciones posibles.//
La desventaja principal es el alto orden de complejidad computacional es decir su tiempo//
de ejecución es muy alto.
\section{conclusion}
Se espera que el proyecto realizado y los conocimientos obtenidos pueda ser útil en proyectos futuros ya sea en la carrera o en la vida laboral ya que la rama de la inteligencia artificial se extiende de manera exponencial,  ya que todo lo relacionado con la tecnología se encuentra influenciado por la inteligencia artificial o diferentes ramas de esta materia.
Como en el caso de este proyecto en donde se utilizo un algoritmo de búsqueda para la// resolución del problema.
